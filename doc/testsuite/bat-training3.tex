\documentclass[11pt]{beamer}

\usepackage{url}
\usepackage{tikz}
%\author{Armijn Hemel}
\title{Using the Binary Analysis Tool - part 3}
\date{}

\begin{document}

\setlength{\parskip}{4pt}

\frame{\titlepage}

\frame{
\frametitle{Subjects}
In this course you will learn:

\begin{itemize}
\item about the scan order in the Binary Analysis Tool
\item to configure the Binary Analysis Tool
\end{itemize}
}

\frame{
\frametitle{Scan order in the Binary Analysis Tool}

\begin{enumerate}
\item the binary is read and file type specific identifiers (from a hardcoded list) are searched for in the binary
\item prerun scans are run to tag files to filter out specific file types later
\item unpacking scans are run to unpack any compressed files or file systems and scans 1 - 3 are run recursively
\item file specific scans are run on all unpacked files
\item aggregate scans are run that act on files in the complete context 
\item post run scans are run for each individual file
\item result files are packed into a result archive
\end{enumerate}
}

\frame{
\frametitle{Identifier search}
The identifier search uses a hardcoded list of identifiers that indicate where a certain file starts or stops. Not all file types have identifiers, but most do.

The locations of identifiers are passed to later scans, which can use this information to work in a more efficient way.
}

\frame{
\frametitle{Prerun scans}
Prerun scans are run to determine the type of the \textit{entire} file and to tag it. Tags can be used by later scans to ignore files: a scan to unpack a file system won't work on a graphics file or a text file.

Some tags that can be set by prerun scans in the current version of BAT are:

\begin{itemize}
\item text
\item binary
\item graphics
\item audio
\item elf
\item compressed
\end{itemize}

plus more to come.
}

\frame{
\frametitle{Unpacking scans}
Unpacking scans try to extract file systems and compressed files, sometimes from a larger binary blob, by ``carving'' it from a larger binary. Currently some 30 file systems, compressed files and media files are supported, including:

\begin{itemize}
\item file systems: Android sparse files, cramfs, ext2/ext3/ext4, ISO9660, JFFS2,
Minix (specific variant of v1), SquashFS (many variants), romfs, YAFFS2 (specific variants)
\item compressed files and executable formats: 7z, ar, ARJ, BASE64, BZIP2,
compressed Flash, CAB, compress, CPIO, EXE (specific compression methods only),
GZIP, InstallShield (old versions), LRZIP, LZIP, LZMA, LZO, RAR, RPM, serialized
Java, TAR, UPX, XZ, ZIP
\item media files: GIF, ICO, PDF, PNG
\end{itemize}
}

\frame{
\frametitle{File specific scans}

\begin{itemize}
\item dynamically linked libraries
\item quick scan for identifiers of several programs (iptables, wireless-tools, etcetera)
\item quick scan for identifiers of several licenses
\item quick scan for identifiers of several collaborative software development sites (sourceforge, github, etcetera)
\item Linux kernel version check
\item BusyBox version check
\item identifier extraction
\item \dots
\end{itemize}
}

\frame{
\frametitle{Aggregate scans}

\begin{itemize}
\item advanced string matching scan using information from a large database extracted from source code
\item finding duplicate files the entire scanned archive
\item checking correctness of declared dependencies in dynamically linked ELF files
\item looking up version information and license information of found results, with optional version pruning
\item generating reports of the results
\item generating pictures of the results
\end{itemize}
}

\frame{
\frametitle{Postrun scans}
Postrun scans are scans that don't modify the files (unpacking, or scanning files), but merely act on results of scans, for example:

\begin{itemize}
\item generating histograms of byte values inside a file
\item generating graphical representations of a file
\item generating hexdump representations of a file
\end{itemize}
}

\frame{
\frametitle{Configuring the Binary Analysis Tool}

The Binary Analysis Tool is highly configurable and uses plugins. These plugins can be enabled and disabled via a configuration file.

The configuration file is in Windows INI format and contains several parts:

\begin{itemize}
\item general configuration
\item configuration directives for ``prerun'' scans
\item configuration directives for ``unpack'' scans
\item configuration directives for ``leaf'' scans
\item configuration directives for ``aggregate'' scans
\item configuration directives for ``postrun'' scans
\end{itemize}

The general configuration is mandatory, the other directives are optional.
}

\frame{
\frametitle{General configuration}
BAT is very configurable. The configuration file lists many options, with descriptions of what they do.

The most important sections to get started quickly are related to:

\begin{itemize}
\item unpacking and multiprocessing
\item database
\item debugging
\end{itemize}
}

\begin{frame}[fragile]
\frametitle{General configuration: unpacking and multiprocessing}

\begin{verbatim}
[batconfig]
multiprocessing = yes
unpackdirectory     = /home/bat/tmp
temporary_unpackdirectory = /ramdisk
configdirectory     = /home/bat/configs
\end{verbatim}

\begin{itemize}
\item \texttt{multiprocessing} is for using multiple threads. If this is not desirable it should be set to \texttt{no}
\item \texttt{unpackdirectory}: set prefix of toplevel directory where data should be unpacked (default: \texttt{/tmp})
\item \texttt{temporary\_unpackdirectory}: set prefix for some scans where temporary files should be created (default: default somewhere in the unpacking directory)
\item \texttt{configdirectory}: path of directory that contain extra configuration files for plugins
\end{itemize}

\end{frame}

\begin{frame}[fragile]
\frametitle{General configuration: database}

\begin{verbatim}
postgresql_user     = bat
postgresql_password = bat
postgresql_db       = bat
postgresql_port     = 5432
postgresql_host     = 127.0.0.1
#usedatabase         = no
\end{verbatim}

\begin{itemize}
\item \texttt{postgresql\_user}: PostgreSQL user to use
\item \texttt{postgresql\_password}: PostgreSQL password
\item \texttt{postgresql\_db}: PostgreSQL database to use
\item \texttt{postgresql\_port}: TCP port to use
\item \texttt{postgresql\_host}: host to use
\end{itemize}

If everything runs on a local machine commenting \texttt{postgresql\_port} and \texttt{postgresql\_host} will force PostgreSQL to use the local socket.
\end{frame}

\begin{frame}[fragile]
\frametitle{General configuration: debugging}
There are two optional configurations related to debugging:

\begin{verbatim}
debug               = yes
debugphases         = unpack:aggregate
\end{verbatim}

\begin{itemize}
\item \texttt{debug}: disable multiprocessing and output debugging information on standard error
\item \texttt{debugphases}: apply \texttt{debug} only to specific scanning phases
\end{itemize}
\end{frame}

\frame{
\frametitle{Mandatory scan configuration options}
The configuration for each type of scan has a few mandatory options:

\begin{itemize}
\item \texttt{type} - \texttt{prerun}, \texttt{unpack}, \texttt{leaf}, \texttt{aggregate} or \texttt{postrun}
\item \texttt{module} - Python module (including package) the scan can be found
\item \texttt{method} - the method for the scan
\item \texttt{enabled} - \texttt{yes} enables a scan, \texttt{no} disables a scan
\end{itemize}
}

\frame{
\frametitle{Optional scan configuration options (1)}

\begin{itemize}
\item \texttt{priority} - scan priority: a higher priority means it is run before scans with a lower priority.
\item \texttt{noscan} - list of tags. Files that are tagged with any of these tags are ignored by this scan.
\item \texttt{scanonly} - list of tags. Only files that are tagged with any of these tags will be scanned. If \texttt{noscan} is also set the intersection of the two will be scanned.
\item \texttt{envvars} - colon separated list of environment variables that will be set in the scan
\item \texttt{magic} - list of specific identifiers this scan is meant for
\end{itemize}
}

\frame{
\frametitle{Optional scan configuration options (2)}

\begin{itemize}
\item \texttt{description} - a human readable description of the scan
\item \texttt{storetype}, \texttt{storetarget} and \texttt{storedir} - used for postrun scans and aggregate scans to describe which files to store in the output, where to store them and where to get them
\item \texttt{cleanup} - used in postrun scans and aggregate scans to tell BAT to clean up any generated file. Used in combination with \texttt{storetype}, \texttt{storetarget} and \texttt{storedir}.
\item \texttt{parallel} - used to indicate whether files should be scanned in parallel. Defaults to \texttt{yes}. If set to \texttt{no} no files will be scanned in parallel for scans of this type.
\item \texttt{setup} - method for running setup code (checking directories, databases, and so on).
\end{itemize}
}

\begin{frame}[fragile]
\frametitle{Prerun scans configuration directive example}

\begin{verbatim}
[checkXML]
type        = prerun
module      = bat.prerun
method      = searchXML
priority    = 2
noscan      = elf:graphics:compressed:font:java:sqlite3:audio:video
description = Check XML validity
enabled     = yes
\end{verbatim}

\end{frame}

\begin{frame}[fragile]
\frametitle{Unpacking scans configuration directive example}

The following example is for unpacking 7z compressed files. The priority is very low, only the identifier for \texttt{7z} is used, and a bunch of file types can safely be ignored by this scan:

\begin{verbatim}
[7z]
type        = unpack
module      = bat.fwunpack
method      = searchUnpack7z
priority    = 1
magic       = 7z
noscan      = text:xml:graphics:pdf:bz2:gzip:lrzip:
              audio:video
description = Unpack 7z compressed files
enabled     = yes
\end{verbatim}

Note: \texttt{noscan} has been split for readability but is actually one line in the configuration file.
\end{frame}

\begin{frame}[fragile]
\frametitle{Aggregate scans configuration directive example}
\begin{verbatim}
[findlibs]
type        = aggregate
module      = bat.findlibs
method      = findlibs
envvars     = BAT_IMAGEDIR=/tmp/images:ELF_SVG=1
noscan      = text:xml:graphics:pdf:audio:video:mp4
enabled     = no
storetarget = images
storedir    = /tmp/images
storetype   = -graph.png:-graph.svg
description = Generate graphs of ELF binary linking
cleanup     = yes
priority    = 5
\end{verbatim}
\end{frame}

\begin{frame}[fragile]
\frametitle{Postrun scans configuration directive example}
\begin{verbatim}
[hexdump]
type        = postrun
module      = bat.generatehexdump
method      = generateHexdump
noscan      = text:xml:graphics:pdf:audio:video:mp4
envvars     = BAT_REPORTDIR=/tmp/images:
              BAT_IMAGE_MAXFILESIZE=100000000
description = Create hexdump output of files
enabled     = no
storetarget = reports
storedir    = /tmp/images
storetype   = -hexdump.gz
\end{verbatim}

Note: \texttt{envvars} has been split for readability but is actually one line in the configuration file.
\end{frame}

\frame{
\frametitle{Output the results of the Binary Analysis Tool}
The raw output of the Binary Analysis Tool is written as a tar archive. The tar archive consists of:

\begin{itemize}
\item full directory tree of unpacked files (if any), if \texttt{outputlite} was specified (default: \texttt{no})
\item gzip compressed Python pickles with the results of the scan
\item (optional) pictures and reports with results of the scan
\end{itemize}

The results can be viewed using the Binary Analysis Tool result viewer.
}

\frame{
\frametitle{Conclusion}
In this course you have learned about:

\begin{itemize}
\item about the scan order in the Binary Analysis Tool
\item to configure the Binary Analysis Tool
\end{itemize}

In the next course we will see how to view results of the Binary Analysis Tool using the special viewer program.
}
\end{document}
